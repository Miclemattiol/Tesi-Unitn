\chapter{Sviluppo (tmp)}
\label{cha:sviluppo}

La progettazione dell'applicazione è iniziata ad aprile durante la mia
esperienza di tirocinio presso l'azienda Level Up S.R.L. di Trento, sotto l'assistenza
del mio tutor aziendale Tommaso Rosi.\\ Il periodo svolto all'interno dell'azienda
è stato fondamentale per la creazione del progetto. Confrontandomi con Tommaso ho
potuto infatti imparare rapidamente a muovermi in maniera autonoma nello sviluppo
dell'applicazione, risolvendo in maniera molto più rapida i problemi che si
manifestavano.

Nel corso del periodo di sviluppo si sono susseguite diverse fasi che possono
essere categorizzate come:
\begin{itemize}
  \item \textbf{Formazione}: Essendo la mia prima esperienza nell'ambito dello sviluppo
    in realtà virtuale, nonostante l'aiuto di Tommaso, è stato necessario affrontare
    un periodo di formazione durante il quale ho potuto apprendere i principali aspetti
    tecnici relativi a questa tecnologia. Questa fase è stata fondamentale per
    la scelta delle tecnologie da utilizzare, permettendomi di imparare le principali
    differenze fra i diversi strumenti di sviluppo (SDK) e motori grafici esistenti.

  \item \textbf{Analisi}: Dovendomi interfacciare contemporaneamente con studenti
    e insegnanti, ho dovuto alternare alle fasi di implementazione, diverse fasi
    di analisi per verificare i singoli bisogni, risolvendo i contrasti presenti
    tra le due categorie.

  \item \textbf{Implementazione}: Questa fase si è suddivisa in diverse sottofasi
    relative alle singole funzionalità che compongono l'applicazione:
    \begin{itemize}
      \item Player RIG

      \item Sistema chiuso

      \item Navigazione

      \item UI

      \item Esperimenti
    \end{itemize}
    Ho deciso di suddividere il lavoro in questi moduli per poter gestire ogni
    aspetto dell'applicazione in maniera indipendente dalle altre fasi, permettendomi
    di concentrarmi su un singolo aspetto alla volta senza preoccuparmi della formazione
    di bug o problemi di compatibilità con le altre parti dell'applicazione.
\end{itemize}

\section{Software e Hardware}
\label{sec:software-hardware}

La prima scelta affrontata è stata il motore grafico. Questa scelta è molto
importante durante la progettazione e lo sviluppo dell'applicazione.\\ Il motore
grafico è il software che permette di astrarre la creazione di ambienti virtuali
permettendo allo sviluppatore di concentrarsi principalmente sullo sviluppo
delle meccaniche dell'applicazione, proprio da qui deriva l'importanza di una scelta
che compatibile con le esigenze di sviluppo e di performance dell'applicazione.\\
I principali motori grafici disponibili sono:

\begin{itemize}
  \item \textbf{Unity3D}: è uno dei motori grafici più utilizzati nello sviluppo
    di applicazioni su spazio bidimensionale e tridimensionale. Presenta una
    curva di apprendimento molto bassa, permettendo a sviluppatori meno esperti di
    imparare più rapidamente lo sviluppo delle applicazioni tramite il linguaggio
    di programmazione C\# . Nonostante la sua semplicità Unity3D rimane un
    motore molto potente e flessibile che permette lo sviluppo di qualsiasi tipo
    di applicazione. Inoltre grazie al suo store di risorse, Unity Asset Store, è
    possibile usufruire di una vasta gamma di risorse utili a velocizzare il processo
    di sviluppo delle applicazioni.

  \item \textbf{Unreal Engine}: è un motore grafico sviluppato da Epic Games,
    utilizzato principalmente nello sviluppo di videogiochi e applicazioni 3D. È
    un motore grafico molto potente e flessibile che vanta una serie di funzionalità
    molto avanzate per permettere lo sviluppo di funzionalità complesse riducendo
    il codice necessario attraverso l'utilizzo di un sistema di programmazione visuale
    chiamato Blueprints, affiancato allo sviluppo tramite il linguaggio di
    programmazione C++. A differenza di Unity3D, Unreal Engine presenta una
    curva di apprendimento molto ripida, rendendolo meno adatto a progetti di dimensioni
    ristrette, o a sviluppatori non esperti.
\end{itemize}

In aggiunta alla scelta del motore grafico è stato necessario scegliere gli
strumenti di sviluppo (SDK) da utilizzare. Questa scelta risulta importante nella
selezione della piattaforma di utilizzo dell'applicazione. In questo tra caso le
opzioni disponibili erano presenti:
\begin{itemize}
  \item \textbf{Meta SDK} (ex Oculus): è uno dei principali SDK per lo sviluppo di
    applicazioni per i dispositivi Meta. Questo SDK permette di sfruttare tutte
    le funzionalità offerte dai dispositivi Meta, permettendo lo sviluppo di applicazioni
    che sfruttino al meglio le funzionalità offerte dai dispositivi.

  \item \textbf{Unity XR interaction Toolkit}: è un toolkit sviluppato da Unity Technologies
    che permette di sviluppare applicazioni per la realtà virtuale compatibili con
    diversi dispositivi. Questo toolkit permette di utilizzare componenti
    standardizzati per lo sviluppo di applicazioni, permettendo di creare applicazioni
    compatibili con diversi dispositivi senza dover riscrivere il codice per ogni
    dispositivo.
\end{itemize}

Siccome l'applicazione dovrà essere eseguita su un dispositivo non
necessariamente collegato ad un computer, ho pensato che i migliori dispositivi per
l'utilizzo dell'applicazione fossero i visori Meta Quest che oltre ad essere indipendenti,
risultano anche essere tra i più acquistati sul mercato. La scelta degli strumenti
di sviluppo è quindi ricaduta sul Meta XR SDK così da poter utilizzare al massimo
le funzionalità dei visori, tra cui in particolare la realtà estesa. Infine per praticità
ho scelto di utilizzare Unity3D come motore grafico per la semplicità di installazione
del Meta XR SDK, e della mia conoscenza pregressa di questo motore.
\section{Formazione}
\label{sec:formazione}

Una folta effettuate le scelte riguardanti le tecnologie da utilizzare, è stato
necessario affrontare un periodo di formazione per apprendere le basi della
realtà virtuale il cui funzionamento non varia in maniera significativa rispetto
allo sviluppo di applicazioni 3D tradizionali, ma che presenza delle differenze non
trascurabili. In particolare i punti fondamentali da affrontare sono stati:
\begin{itemize}
  \item \textbf{La render pipeline}: la render pipeline è il processo che permette
    di trasformare i modelli 3D in immagini 2D. La scelta di questo si è
    rivelata molto importante in quanto la pipeline built-in di Unity3D non è
    compatibile con gli asset di Meta che mostravano quindi una colorazione
    fucsia, rendendo necessario l'utilizzo della render pipeline di Universal Render
    Pipeline (URP).

  \item \textbf{Le interazioni}: in Unity, le interazioni rappresentano tutte le
    azioni eseguibili dall'utente. Queste azioni possono essere di vario tipo,
    ad esempio il movimento, lo spostamento di oggetti, la pressione di un
    bottone o l'utilizzo di una gesture. Esistono strutture apposite per la gestione
    di ognuna di queste azioni, che devono essere aggiunte e configurate in
    maniera corretta per permettere il corretto funzionamento del sistema.

  \item \textbf{La fisica}: la fisica è l'aspetto fondamentale dell'applicazione,
    in quanto tale l'utente deve avere la possibilità di controllarla in maniera
    completa. Per questo motivo è stato necessario studiare la gestione delle costanti
    fisiche del sistema, permettendo all'utente di impostare parametri
    differenti per diversi esperimenti coesistenti.
\end{itemize}

Lo studio di queste funzionalità non è stato particolarmente complesso, ad eccezione
delle funzionalità del Meta XR SDK che si sono rivelate più complesse a causa della
scarsa documentazione disponibile online. Per questo motivo è stato necessario
fare affidamento su fonti esterne come forum e tutorial che mi permettessero di
apprendere il funzionamento delle funzionalità più importanti.

\textbf{TODO: Riscrivere impersonale}
\section{Sviluppo Player RIG}
\label{sec:player-rig}

Il \textit{Player RIG} è la struttura che definisce l'utente, il suo aspetto e le
sue capacità all'interno dell'ambiente. In particolare questo è composto da un
controller degli input del visore, due controller degli input relative ai controller % TODO: Sostituire la ripetizione
usati dall'utente, e da un insieme di interazioni possibili.
\section{Sviluppo sistema chiuso}
\label{sec:sistema-chiuso}

I sistemi chiusi sono il concetto fondamentale dell'applicazione. Questi
permettono ai singoli esperimenti di svolgersi all'interno di uno spazio
dedicato da cui sono esclusi tutti gli agenti esterni come gravità o altre forze.

L'interazione con i sistemi chiusi viene gestita da un apposito gestore
responsabile della creazione, dell'inizializzazione e dell'interrogazione dei singoli
sistemi così da mostrare i dati in maniera corretta e controllata.
\section{Sviluppo navigazione}
\label{sec:navigazione}

La navigazione è gestita tramite appositi gesti manuali che permettono all'utente
in maniera comoda di visualizzare i menu su richiesta senza la necessità di controllare
il loro posizionamento. Questi infatti sono ancorati alla mano dell'utente, e saranno
richiamabili pizzicando le dita della mano sinistra.

I menu di gestione degli esperimenti saranno galleggianti, per permettere all'utente
di usufruirne senza la necessità di mantenere una determinata posizione, e saranno
richiamabili in qualsiasi momento nel caso non fosse possibile l'uso nel precedente
luogo di invocazione.
\section{UI}
\label{sec:ui}

Lo studio delle interfacce utente si è basato principalmente sul tipo di utente mirato.
In particolare è stato necessario rendere l'uso semplificato per favorire la
classe degli insegnanti, aggiungendo contemporaneamente metodologie d'uso moderne
che permettessero ai giovani studenti un uso più rapido e intuitivo.