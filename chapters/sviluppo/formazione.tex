\section{Formazione}
\label{sec:formazione}

Una folta effettuate le scelte riguardanti le tecnologie da utilizzare, è stato
necessario affrontare un periodo di formazione per apprendere le basi della
realtà virtuale il cui funzionamento non varia in maniera significativa rispetto
allo sviluppo di applicazioni 3D tradizionali, ma che presenza delle differenze non
trascurabili. In particolare i punti fondamentali da affrontare sono stati:
\begin{itemize}
  \item \textbf{La render pipeline}: la render pipeline è il processo che permette
    di trasformare i modelli 3D in immagini 2D. La scelta di questo si è
    rivelata molto importante in quanto la pipeline built-in di Unity3D non è
    compatibile con gli asset di Meta che mostravano quindi una colorazione
    fucsia, rendendo necessario l'utilizzo della render pipeline di Universal Render
    Pipeline (URP).

  \item \textbf{Le interazioni}: in Unity, le interazioni rappresentano tutte le
    azioni eseguibili dall'utente. Queste azioni possono essere di vario tipo,
    ad esempio il movimento, lo spostamento di oggetti, la pressione di un
    bottone o l'utilizzo di una gesture. Esistono strutture apposite per la gestione
    di ognuna di queste azioni, che devono essere aggiunte e configurate in
    maniera corretta per permettere il corretto funzionamento del sistema.

  \item \textbf{La fisica}: la fisica è l'aspetto fondamentale dell'applicazione,
    in quanto tale l'utente deve avere la possibilità di controllarla in maniera
    completa. Per questo motivo è stato necessario studiare la gestione delle costanti
    fisiche del sistema, permettendo all'utente di impostare parametri
    differenti per diversi esperimenti coesistenti.
\end{itemize}

Lo studio di queste funzionalità non è stato particolarmente complesso, ad eccezione
delle funzionalità del Meta XR SDK che si sono rivelate più complesse a causa della
scarsa documentazione disponibile online. Per questo motivo è stato necessario
fare affidamento su fonti esterne come forum e tutorial che mi permettessero di
apprendere il funzionamento delle funzionalità più importanti.

\textbf{TODO: Riscrivere impersonale}