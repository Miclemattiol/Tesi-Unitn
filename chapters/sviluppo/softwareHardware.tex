\section{Software e Hardware}
\label{sec:software-hardware}

La prima scelta affrontata è stata il motore grafico. Questa scelta è molto
importante durante la progettazione e lo sviluppo dell'applicazione.\\ Il motore
grafico è il software che permette di astrarre la creazione di ambienti virtuali
permettendo allo sviluppatore di concentrarsi principalmente sullo sviluppo
delle meccaniche dell'applicazione, proprio da qui deriva l'importanza di una scelta
che compatibile con le esigenze di sviluppo e di performance dell'applicazione.\\
I principali motori grafici disponibili sono:

\begin{itemize}
  \item \textbf{Unity3D}: è uno dei motori grafici più utilizzati nello sviluppo
    di applicazioni su spazio bidimensionale e tridimensionale. Presenta una
    curva di apprendimento molto bassa, permettendo a sviluppatori meno esperti di
    imparare più rapidamente lo sviluppo delle applicazioni tramite il linguaggio
    di programmazione C\# . Nonostante la sua semplicità Unity3D rimane un
    motore molto potente e flessibile che permette lo sviluppo di qualsiasi tipo
    di applicazione. Inoltre grazie al suo store di risorse, Unity Asset Store, è
    possibile usufruire di una vasta gamma di risorse utili a velocizzare il processo
    di sviluppo delle applicazioni.

  \item \textbf{Unreal Engine}: è un motore grafico sviluppato da Epic Games,
    utilizzato principalmente nello sviluppo di videogiochi e applicazioni 3D. È
    un motore grafico molto potente e flessibile che vanta una serie di funzionalità
    molto avanzate per permettere lo sviluppo di funzionalità complesse riducendo
    il codice necessario attraverso l'utilizzo di un sistema di programmazione visuale
    chiamato Blueprints, affiancato allo sviluppo tramite il linguaggio di
    programmazione C++. A differenza di Unity3D, Unreal Engine presenta una
    curva di apprendimento molto ripida, rendendolo meno adatto a progetti di dimensioni
    ristrette, o a sviluppatori non esperti.
\end{itemize}

In aggiunta alla scelta del motore grafico è stato necessario scegliere gli
strumenti di sviluppo (SDK) da utilizzare. Questa scelta risulta importante nella
selezione della piattaforma di utilizzo dell'applicazione. In questo tra caso le
opzioni disponibili erano presenti:
\begin{itemize}
  \item \textbf{Meta SDK} (ex Oculus): è uno dei principali SDK per lo sviluppo di
    applicazioni per i dispositivi Meta. Questo SDK permette di sfruttare tutte
    le funzionalità offerte dai dispositivi Meta, permettendo lo sviluppo di applicazioni
    che sfruttino al meglio le funzionalità offerte dai dispositivi.

  \item \textbf{Unity XR interaction Toolkit}: è un toolkit sviluppato da Unity Technologies
    che permette di sviluppare applicazioni per la realtà virtuale compatibili con
    diversi dispositivi. Questo toolkit permette di utilizzare componenti
    standardizzati per lo sviluppo di applicazioni, permettendo di creare applicazioni
    compatibili con diversi dispositivi senza dover riscrivere il codice per ogni
    dispositivo.
\end{itemize}

Siccome l'applicazione dovrà essere eseguita su un dispositivo non
necessariamente collegato ad un computer, ho pensato che i migliori dispositivi per
l'utilizzo dell'applicazione fossero i visori Meta Quest che oltre ad essere indipendenti,
risultano anche essere tra i più acquistati sul mercato. La scelta degli strumenti
di sviluppo è quindi ricaduta sul Meta XR SDK così da poter utilizzare al massimo
le funzionalità dei visori, tra cui in particolare la realtà estesa. Infine per praticità
ho scelto di utilizzare Unity3D come motore grafico per la semplicità di installazione
del Meta XR SDK, e della mia conoscenza pregressa di questo motore.