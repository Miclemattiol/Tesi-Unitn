\chapter{Introduzione}
\label{cha:introduzione}

L'uso della tecnologia nelle scuole ha cambiato molto l'apprendimento e l'interazione
degli studenti con i contenuti didattici, cambiamento che l'insegnamento tradizionale
non ha ancora superato a causa delle sue limitazioni, come la difficoltà di
rappresentare fenomeni complessi e la scarsa interattività. In questo contesto la
realtà virtuale si afferma come una tecnologia in grado di migliorare l'esperienza
di apprendimento.

L'uso dei visori nelle scuole permette agli studenti di immergersi maggiormente nello
studio, permettendo attraverso strategie di \textit{gamification} di rendere l'apprendimento
più coinvolgente e divertente, migliorando allo stesso tempo la comprensione e l'attenzione
nelle materie più complesse

Il progetto verte quindi sulla realizzazione di un ambiente virtuale che supporta
l'insegnamento scolastico permettendo agli studenti di interagire con gli insegnamenti
in maniera maggiormente immersiva.

Il focus del progetto verte sulla fisica. È stato nominato \textit{PhXR} (Physics
in eXtended Reality) e permette di sperimentare esperimenti fisici in maniera immersiva
all'interno di un ambiente virtuale totalmente controllabile. Infatti per ogni
esperimento è possibile controllare le costanti fisiche permettendo quindi di
visualizzare un fenomeno escludendo le possibili variazioni fisiche date da
agenti indesiderati come ad esempio la forza d'attrito.

% TODO:
\textbf{TODO: Riferimenti a ricerche che trattano l'uso dei visori nell'insegnamento}