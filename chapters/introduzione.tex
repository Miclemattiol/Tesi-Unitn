\chapter{Introduzione}
\label{cha:introduzione}

\section{Citazioni}
\label{cha:examples_citations}

\texttt{@misc}\cite{misc} \\ %
\texttt{@book}\cite{book} \\ %
\texttt{@conference}\cite{conference} \\ %
\texttt{@article}\cite{article}

Il progetto PhXR nasce dalla mia precedente esperienza da insegnante all'istituto
superiore Buonarroti di Trento durante la quale ho avuto la possibilità di osservare
in prima persona le tecnologie in possesso delle scuole (spesso nemmeno
utilizzate) e le difficoltà riscontrate dagli studenti nell'apprendimento delle materie
scientifiche. Da qui nasce l'idea di sfruttare la realtà virtuale come strumento
per semplificare e migliorare l'apprendimento attraverso una strategia di
\textit{gamification}.

\section{Obiettivi}
\label{sec:introduzione_obiettivi}

L'obiettivo principale del progetto è quindi quello di facilitare l'apprendimento
delle materie scientifiche attraverso l'uso della realtà virtuale, sfruttando contemporaneamente
questi dispositivi spesso inutilizzati dagli istituti scolastici. L'applicazione
sviluppata prende come esempio la fisica, ma il progetto è facilmente estendibile
ad altre materie scientifiche come la chimica, la biologia, la matematica, ecc.

% L'obiettivo principale del progetto

\section{Uso della realtà virtuale}
\label{sec:introduzione_uso_realtà_virtuale}

La realtà virtuale permette la creazione di un ambiente virtuale in cui l'utente
cosciente dell'astrazione può sentirsi immerso. Questo permette di osservare un evento
simulato mantenendo quella sensazione di realtà che permette di comprendere meglio
il fenomeno. Essendo questi dispositivi molto attraenti agli occhi degli
studenti, l'uso dei visori permette inoltre di stimolare l'attenzione e l'interesse
degli studenti.

\section{Tecnologie utilizzate}
\label{sec:introduzione_tecnologie_utilizzate}

Il progetto è stato sviluppato utilizzando il motore grafico Unity, uno dei più utilizzati
per la creazione di videogiochi e applicazioni su spazi tridimensionali. La
scelta è dovuta alla presenza del Meta SDK che permette di sviluppare
applicazioni XR installabili direttamente su visori Meta, rimuovendo la necessità
di connessione ad un pc.
