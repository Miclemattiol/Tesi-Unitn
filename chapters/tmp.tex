\chapter{Introduzione}
\label{cha:introduzione}

L'uso della tecnologia nell'istruzione ha trasformato profondamente il modo in
cui gli studenti apprendono e interagiscono con i contenuti didattici. Tuttavia,
l'insegnamento tradizionale presenta ancora diverse limitazioni, tra cui la difficoltà
di rappresentare fenomeni complessi e la scarsa interattività di alcuni metodi.
In questo contesto, la realtà virtuale (\emph{Virtual Reality, VR}) si è
affermata come una tecnologia innovativa in grado di offrire ambienti immersivi
e altamente interattivi, potenziando l'esperienza di apprendimento.

Questa tesi presenta lo sviluppo di \emph{PhXR}, un progetto che utilizza la realtà
virtuale per creare un laboratorio didattico interattivo. Sebbene inizialmente
focalizzato sull'insegnamento della fisica, il progetto è concepito come un prototipo
per un sistema più ampio, in grado di supportare l’apprendimento di molteplici
discipline scolastiche, incluse quelle non scientifiche. Attraverso la
manipolazione diretta degli elementi della scena e la possibilità di configurare
gli esperimenti in modo personalizzato, \emph{PhXR} si propone di superare i
limiti dell’insegnamento tradizionale, offrendo agli studenti un ambiente di apprendimento
più coinvolgente e intuitivo.

\section{Motivazione e contesto}
\label{sec:introduzione_motivazione}

L’educazione tradizionale si basa su modelli didattici consolidati, ma spesso incontra
difficoltà nell’adattarsi alle esigenze di un mondo in continua evoluzione. Le
lezioni frontali e gli esperimenti di laboratorio fisici, per quanto utili,
possono risultare limitanti a causa di fattori come costi elevati, sicurezza,
disponibilità di materiali e impossibilità di riprodurre condizioni ideali per
alcuni fenomeni.

La realtà virtuale offre un'alternativa efficace, permettendo di ricreare esperimenti
e simulazioni in un ambiente digitale interattivo. Diversi studi hanno
dimostrato come l’uso della \emph{VR} possa migliorare la comprensione dei concetti
complessi e aumentare il coinvolgimento degli studenti. Il progetto \emph{PhXR} nasce
con l'obiettivo di esplorare questo potenziale, sviluppando un sistema che, oltre
alla fisica, possa essere esteso in futuro a molte altre discipline.

\section{Obiettivi del progetto}
\label{sec:introduzione_obiettivi}

Il progetto \emph{PhXR} si pone come obiettivo principale lo sviluppo di un
ambiente virtuale per il supporto all’insegnamento scolastico, inizialmente in ambito
scientifico, ma con la possibilità di espandersi ad altre materie. In particolare,
il sistema è progettato per garantire:

\begin{itemize}
  \item \textbf{Un laboratorio virtuale interattivo:} creare un ambiente
    immersivo in cui gli studenti possano eseguire esperimenti e simulazioni in
    modo pratico e coinvolgente.

  \item \textbf{Un'interfaccia intuitiva basata su gesture:} permettere l'interazione
    naturale con l'ambiente tramite movimenti delle mani, eliminando la necessità
    di comandi complessi.

  \item \textbf{Un'architettura modulare e scalabile:} progettare il sistema in
    modo che sia facilmente estendibile a nuove discipline oltre la fisica.

  \item \textbf{Esperienza utente fluida e senza interruzioni:} separare la
    scena del \emph{Player RIG} da quella degli esperimenti per garantire un'esperienza
    immersiva senza tempi di caricamento percepibili.

  \item \textbf{Compatibilità con dispositivi VR moderni:} sviluppare il sistema
    utilizzando \emph{Unity} e il framework \emph{Meta SDK}, ottimizzandolo per il
    visore \emph{Meta Quest 3}.
\end{itemize}

\section{Struttura della tesi}
\label{sec:introduzione_struttura}

Il resto della tesi è organizzato come segue:

\begin{itemize}
  \item \textbf{Capitolo~\ref{cha:stato_arte}:} introduce lo stato dell'arte
    delle applicazioni VR per l’istruzione, analizzando i principali sistemi esistenti
    e le loro caratteristiche.

  \item \textbf{Capitolo~\ref{cha:architettura}:} descrive l’architettura
    software e le scelte di progettazione adottate nello sviluppo di \emph{PhXR}.

  \item \textbf{Capitolo~\ref{cha:implementazione}:} approfondisce
    l’implementazione tecnica dell’applicazione, illustrando le tecnologie e le metodologie
    utilizzate.

  \item \textbf{Capitolo~\ref{cha:valutazione}:} presenta i test eseguiti per
    valutare le prestazioni e l’usabilità del sistema, discutendone i risultati.

  \item \textbf{Capitolo~\ref{cha:conclusioni}:} conclude la tesi riassumendo i
    contributi principali del lavoro e suggerendo possibili sviluppi futuri.
\end{itemize}

Il progetto \emph{PhXR} rappresenta un passo verso l’integrazione della realtà
virtuale nell'istruzione, con il potenziale di trasformare radicalmente il modo in
cui gli studenti apprendono e interagiscono con i contenuti didattici.